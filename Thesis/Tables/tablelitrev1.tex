\renewcommand{\arraystretch}{2.0}
\newcolumntype{P}{>{\raggedright\arraybackslash}p{0.15\textwidth}}
\newcolumntype{Q}{>{\raggedright\arraybackslash}p{0.24\textwidth}}
\tiny
\begin{longtable}{PQQQ}
\label{tabledataextraction1} \\
\caption{Summary of literature review studies (doctor participants)} \\
\hline
\textbf{Study}&\textbf{Tsai 2003}~\citep{tsai_computer_2003}&\textbf{Massel 2003}~\citep{massel_observer_2003}&\textbf{Goodacre 2001}~\citep{goodacre_computer_2001} \\
\hline
\endfirsthead
\multicolumn{4}{l}
{\tablename\ \thetable\ -- \textit{Continued from previous page}} \\
\hline
\textbf{Study}&\textbf{Tsai 2003}&\textbf{Massel 2003}&\textbf{Goodacre 2001} \\
\hline
\endhead
\hline \multicolumn{4}{r}{\textit{Continued on next page}} \\
\endfoot
\hline
\endlastfoot
Population&30 doctors (internal medical residents) in second or third year of training.&9 doctors. 3 medical residents, 3 cardiology fellows and 3 consultant cardiologists.&10 doctors, all senior house officers.\\
Intervention&23 ECGs (11 in group A, 12 in group B) with computer interpretation message.&75 ECGs with typical history, checklist and computer interpretation.&25 ECGs with computer interpretation message.\\
Comparison&23 ECGs (11 in group A, 12 in group B) with computer interpretation message hidden.&75 ECGs with atypical history, no checklist and no computer interpretation.&25 ECGs without computer interpretation message.\\
Outcomes&Interpretative accuracy of the medicine residents. Secondary outcome measure: the effect of incorrect computer interpretation.&Intra- and inter-observer variability and bias measurements.&Proportion of major errors missed in each group. Secondary outcome measure: number of completely correct ECGs, without major or minor errors.\\
Site(s)&US university department of medicine.&Single tertiary care centre in Canada.&Single emergency department in a UK teaching hospital.\\
Results&Without computer interpretation, accuracy 48.9\% (95\% CI, 45.0-52.8\%). With computer interpretation, 55.4\% (95\% CI, 51.9-58.9\%; p$<$0.0001). When the correct computer interpretation included, accuracy 68.1\% (95\% CI, 63.2-72.7\%; p$<$0.0001).  Participants wrongly agreed with incorrect computer interpretation more often when visible 67.7\% (95\% CI, 57.2-76.7\%) than when it was not 34.6\% (95\% CI, 23.8-47.3\%; p$<$0.0001).&When all doctors considered as a group, improvement in inter-observer ECG interpretation when computer message provided (p=0.0001).  Medical residents biased by computerised ECG (p$<$0.001) and less likely to recommend thrombolysis.&Major errors found in 46\slash 250 (18.4\%) ECG interpretations made by SHOs with computer interpretation visible. 56\slash 250 (22.4\%) major errors found in interpretations without computer message visible. No evidence of relationship between computer interpretation use and major errors by SHOs.\\
Notes&Computer algorithm not identified. ECGs not restricted to STEMI.&GE Marquette 12SL algorithm used. ECGs were not restricted to STEMI.&GE Marquette 12SL algorithm used. ECGs were not restricted to STEMI.\\


\end{longtable}
\renewcommand{\arraystretch}{1}
\normalsize