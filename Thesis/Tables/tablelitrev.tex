\renewcommand{\arraystretch}{2.0}
\newcolumntype{L}{>{\raggedright\arraybackslash}p{0.15\textwidth}}
\newcolumntype{A}{>{\raggedright\arraybackslash}p{0.24\textwidth}}
\tiny
\begin{longtable}{LAAA}
\label{tabledataextraction} \\
\caption{Summary of literature review studies (paramedic participants)} \\
\hline
\textbf{Study}&\textbf{Cantor 2012}~\citep{cantor_prehospital_2012}&\textbf{Selker 2011}~\citep{selker_emergency_2011}&\textbf{Ting 2009}~\citep{ting_abstract_2009}\\
\hline
\endfirsthead
\multicolumn{3}{l}
{\tablename\ \thetable\ -- \textit{Continued from previous page}} \\
\hline
\textbf{Study}&\textbf{Cantor 2012}&\textbf{Selker 2011}&\textbf{Ting 2009} \\
\hline
\endhead
\hline \multicolumn{4}{r}{\textit{Continued on next page}} \\
\endfoot
\hline
\endlastfoot
Population&134 consecutive patients with suspected STEMI taken for pPCI.&437 patients over three phases.&2,007 patients but only 54 patients with suspected STEMI.\\
Intervention&New protocol for primary care paramedics. Computer interpretation of 12-lead ECG (GE Marquette 12SL).&Computer interpretation using ACI-TIPI and TPI diagnostic tools. Usual computer interpretation message provided provided (GE Marquette 12SL).&Prehospital ECG protocol. Computer interpretation message provided (GE Marquette 12SL).\\
Comparison&Blinded doctor interpretation of prehospital 12-lead ECG. Final diagnosis confirmed with angiography and cardiac biomarkers.&Blinded doctor with access to patients clinical records, ECGs and cardiac biomarkers.&Diagnosis determined by angiography.\\
Outcomes&Accuracy of STEMI recognition, complications during transfer and treatment times.&Percentage of true and false positive patients identified by paramedics as having ACS or STEMI.&Accuracy of STEMI recognition by paramedics.\\
EMS system, skill level and training&Single site in Canada.  Primary care paramedics in study received 4 hours training on 12-lead ECG STEMI recognition.&11 sites throughout USA.  NAEMT certified paramedics received 4 hours training in ACS and STEMI recognition and ACI-TIPI and TPI tools.&Single site in USA.  67 NAEMT certified paramedics received 3 hours training on protocol, ECG acquisition and interpretation.\\
Results&Doctor agreed with paramedic 121\slash 134 (90\%) participants. Final diagnosis: STEMI 106\slash 134, false positive 28\slash 134. 11\slash 28 false positives could have been excluded if only computer interpretation utilised. 8 true STEMIs missed by computer interpretation.&Comparison between phase 1 and 2: STEMI identification increased from 40.8\% to 68.4\% (p$<$0.01). Retrospective analysis of ACI-TIPI and TPI gave true positive rate for STEMI as 73\%.&Prehospital recognition of STEMI: sensitivity 48.0\%, specificity 99.6\%, Positive predictive value 86.7\%, Negative predictive value 97.4\%. False negatives: 57\% due to inaccurate computer interpretation, 21\% due to cardiac arrest where no ECG recorded.\\
Notes&Primary care paramedics are equivalent to EMT-B in USA and ambulance technicians in the UK.&Training for new (to study) paramedics reduced to 1.5 hours in phase 3. Only study which utilised ACI-TIPI and TPI tools. Analysis of phases retrospective. Patients not required to have chest pain.&Only post-implementation data usable. No record of missed STEMIs not using protocol. Not clear what demographic data recorded for patients who were not in the protocol.\\
\end{longtable}
\renewcommand{\arraystretch}{1}
\normalsize