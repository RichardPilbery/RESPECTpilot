\chapter{RESPECT study and pilot}
\label{respectstudyandpilot}

 \lhead{\emph{RESPECT study and pilot}} % Chapter title for thesis template 

Given the doubt surrounding the effect of computer interpretation on paramedic accuracy in recognising STEMI, and the importance of minimising false positives and negatives, this study will focus on the effect that computer interpretation has on a paramedic's accurate recognition of STEMI. From the literature review, it is clear that the study design needs to take account of clustering. However, in the absence of existing studies to provide insight into intra-class correlation coefficients to determine design effects, a pilot study is necessary. In addition, the pilot provides an opportunity to test a customised, web-based assessment tool, which will record the paramedic participant's ability to correctly diagnose STEMI and exclude STEMI-mimics, both with and without a computer message, using a randomised crossover design~\citep{arain_what_2010}. 

\section{Aims and objectives}
\label{aimsandobjectives}

Pilot study aims: 

\begin{itemize}
\item To create and test a web-based assessment tool, suitable to address the main study aim

\item To determine the feasibility of the main study.

\end{itemize}

 \clearpage 

Pilot study objectives:

\begin{itemize}
\item Source at least 48 ECGs, with a proportional mix of STEMI and STEMI-mimic patterns, all with computer diagnostic messages

\item Obtain a gold-standard diagnosis of each ECG

\item Create custom database-driven website to administer the study

\item Conduct the pilot study to:

\begin{itemize}
\item Determine recruitment rates and identify incentives to minimise attrition in the main study

\item Check the randomisation procedure

\item Test that the assessment tool works correctly

\item Obtain preliminary estimates of the accuracy, sensitivity and specificity of paramedic's interpretation, to determine whether it is appropriate to conduct the main study

\item Estimate intra-class correlation coefficients for participants and ECGs, and the number of discordant proportions, in order to provide guidance in determining the sample size for an appropriately powered main study

\item Construct a generalised linear model to determine the odds ratios relating to paramedics accuracy in recognising STEMI, taking into account the clustering of participant responses and ECG.

\end{itemize}

\end{itemize}

Main study aim: To examine the effect of computer interpretation messages printed on ECGs on the accuracy of paramedics’ recognition of STEMI.

Objectives:

\begin{itemize}
\item Estimate an appropriate sample size to ensure the study is adequately powered, taking into account the clustering of data

\item Recruit paramedics from Yorkshire Ambulance Service to take part in the study

\item Obtain precise estimates of the accuracy, sensitivity and specificity of paramedic's interpretation

\item Estimate the effect of computer-generated messages on paramedic interpretation (overall and
stratified by correct and incorrect computer interpretation)

\item Disseminate the results.

\end{itemize}

\section{Value of this research}
\label{valueofthisresearch}

The results from this pilot study will demonstrate whether an online assessment tool can appropriately test paramedics' accuracy in STEMI recognition. The estimation of intra-class correlation coefficients will help determine appropriate sample sizes for an adequately powered, main study. In addition, the publication of the pilot data will also assist other researchers conducting studies in this area, given the paucity of published studies relating to this topic.

Assuming that the main study is feasible, the results will provide information about the effect that computer interpretation messages has on paramedics' recognition of STEMI. This should assist ambulance services determine whether computer interpretation should continue to be provided to their paramedics and provide insight into the current interpretation accuracy of UK paramedics. 
